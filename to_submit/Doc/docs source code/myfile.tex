\documentclass{beamer}
\usepackage{lmodern}
\usepackage{amsmath,amsfonts,amssymb}
% \RequirePackage{mathpazo} % Widely available alternative to Garamond
% \RequirePackage[scaled]{helvet}
\RequirePackage{url}
\RequirePackage[colorlinks=true, allcolors=blue]{hyperref}
\usepackage{ragged2e}
\usepackage{lmodern}
% Input encoding
\usepackage[utf8]{inputenc}
% Polish language support
\usepackage[polish]{babel}
% Other packages
\usepackage{amsmath,amsfonts,amssymb}
\usepackage{graphicx}
\usepackage[colorlinks=true, allcolors=blue]{hyperref}
% other packages
\usepackage{latexsym,amsmath,xcolor,multicol,booktabs,calligra}
\usepackage{biblatex}
\addbibresource{ref.bib}
\setbeamertemplate{headline}{} 

\usepackage{graphicx,pstricks,listings,stackengine}
\author{\href{}{Piotr Gugnowski, Dawid Kozaczuk, Karol Gutkowski, Kamil Kamiński}}
\institute{\href{}{Teoria Algorytmów i Obliczeń}}
\title{Projekt na temat algorytmów grafowych}
\date{}
\usepackage{JYY_Beamer}
% defs
\def\cmd#1{\texttt{\color{red}\footnotesize $\backslash$#1}}
%\def\env#1{\texttt{\color{blue}\footnotesize #1}}
\definecolor{deepblue}{rgb}{0.3,0.1,0.5}
\definecolor{deepred}{rgb}{0.6,0,0}
\definecolor{deepgreen}{rgb}{0,0.5,0}
\definecolor{halfgray}{gray}{0.55}
\definecolor{dufecolor}{RGB}{12,60,100}
\lstset{
    basicstyle=\ttfamily\small,
    keywordstyle=\bfseries\color{deepblue},
    emphstyle=\ttfamily\color{deepred},   % Custom highlighting style
    stringstyle=\color{deepgreen},
    numbers=left,
    numberstyle=\small\color{halfgray},
    rulesepcolor=\color{red!20!green!20!blue!20},
    frame=shadowbox,
}

\AtBeginSection[]{}

\begin{document}

\begin{frame}
  \titlepage
\end{frame}

\begin{frame}{Spis treści}
\small % Reduce font size
    \tableofcontents % Show only sections
\end{frame}

\section{Opracowanie pojęć}
\begin{frame}{Opracowanie pojęć}
\begin{block}{Rozmiar grafu}
    Przyjmujemy, że rozmiar grafu to ilość krawędzi w grafie.    
\end{block}
\end{frame}

\begin{frame}{Opracowanie pojęć}
\begin{block}{Maksymalny cykl w grafie}
    //TODO   
\end{block}
\end{frame}

\begin{frame}{Opracowanie pojęć}
\begin{block}{Minimalne rozszerzenie grafu do grafu zawierającego cykl Hamiltona}
Cykl Hamiltona to taki cykl w grafie, który zawiera każdy wierzchołek z grafu dokładnie raz.  \\
    Mówiąc o minimalnym rozszerzeniu mamy na myśli rozszerzanie grafu do grafu zawierającego cykl Hamiltona. Przyjmujemy, że minimalne rozszerzenie to te, które zawiera najmniejszą ilość krawędzi, które należy dodać aby graf zawierał skierowany cykl Hamiltona.   
\end{block}
\end{frame}

\begin{frame}{Opracowanie pojęć}
\begin{block}{Liczba cykli Hamiltona}
    
    Przymujemy, że liczbą cykli Hamiltona będziemy nawyzwać liczbę wszystkich możliwych unikalnych skierowanych cykli Hamiltona w grafie to znaczy, dla których kolejność odwiedzanych wierzchołków jest różna.    
\end{block}
\end{frame}

\section{Algorytm 1}
\begin{frame}
  \centering
  \vspace{2cm} % Można dostosować odstęp w zależności od potrzeby
  \Huge \textbf{Znajdowanie minimalnego rozszerzenia do grafu zawierającego cykl Hamiltona} \\
  \vspace{1cm}
  \Large oraz liczbę cykli Hamiltona w rozszerzonym grafie
\end{frame}


\begin{frame}
\frametitle{Opis algorytmu}
\begin{block}{Znajdowanie minimalnego rozszerzenia grafu}
\justifying
Algorytm polega na wielokrotnym wykonaniu podążania losową ścieżką po grafie. \\ 
\begin{itemize}
    \item Najpierw wybierany jest losowy wierzchołek. \\ 
    \item Następnie wybierany jest losowy nieodwiedzony sąsiad wybranego wierzchołka po czym następuje przejście do tego sąsiada. \\ 
    \item Przejścia są powtarzane do momentu, aż w trakcie działania procedury znajdzie się wierzchołek, który nie ma krawędzi wychodzących do nieodwiedzonego wierzchołka. Wtedy następuje dodanie krawędzi. \\ 
    \item Krawędź pozwala przejść dalej do nieodwiedzonego wierzchołka z najmniejszą ilością krawędzi wchodzących z nieodwiedzonych do tej pory wierzchołków. \\ 
\end{itemize}
\end{block}
\end{frame}

\begin{frame}
\frametitle{Opis algorytmu}
\begin{block}{Znajdowanie minimalnego rozszerzenia grafu cd.}
\justifying
\begin{itemize}
    \item Po takim przejściu następuje kontynuowanie podążanie losową ścieżką. \\ 
    \item Gdy odwiedzone zostaną wszystkie wierzchołki następuje przejście do początkowego wierzchołka. Jeśli brakuje krawędzi to jest dodana. \\ 
    \item Rozszerzenie grafu to zbiór dodanych krawędzi podczas procedury budowania ścieżki. \\ 
    \item Wybierany jest najmniej liczby zbiór spośród każdej próby budowania ścieżki. \\
\end{itemize}
\end{block}
\end{frame}

\begin{frame}
\frametitle{Opis algorytmu}
\begin{block}{Znajdowanie liczby cykli Hamiltona w rozszerzonym grafie}
\justifying
Algorytm polega na zliczaniu ścieżek, którymi udało się przejść podczas próby przejścia przez wszystkie wierzchołki dokładnie raz oraz takich, dla których istnieje krawędź w grafie od ostatniego do pierwszego wierzchołka. Szukanie ścieżki jest wykonywane wielokrotnie, każda unikatowa ścieżka jest zapisywana. Na początku wiadomo o jednym cyklu Hamiltona, tym którym podążano przy tworzeniu najmniejszego rozszerzenia do grafu zawierającego cykl Hamiltona.
\end{block}
\end{frame}

\begin{frame}
\frametitle{Opis algorytmu}
\begin{block}{Znajdowanie liczby cykli Hamiltona w rozszerzonym grafie: bez rozszerzenia}
\justifying
\begin{itemize}
    \item Jeśli grafu nie trzeba było rozszerzać, aby zawierał cykl Hamiltona każdy punkt rozpoczęcia budowania ścieżki jest równie dobry, dlatego też każda ścieżka zaczyna się od wierzchołka o indeksie 0. \\
    \item Wykonywana jest procedura podążania losową ścieżką. \\
    \item W przypadku przejścia do wierzchołka, z którego nie można już przejść do nieodwiedzonego wierzchołka sprawdzane jest czy wszystkie wierzchołki zostały odwiedzone i czy istneje krawędź w grafie z ostatniego do pierwszego wierzchołka. Jeśli warunki są spełnione to cykl jest dodawany do zbioru unikalnych cykli. \\
\end{itemize}
\end{block}
\end{frame}

\begin{frame}
\frametitle{Opis algorytmu}
\begin{block}{Znajdowanie liczby cykli Hamiltona w rozszerzonym grafie: z rozszerzeniem}
\justifying
Jeśli graf trzeba było rozszerzyć to jedna z dodanych krawędzi była niezbędna, aby w ogóle jakikolwiek cykl Hamiltona istniał w grafie. Dlatego też opłaca się rozpoczynać budowę ścieżek od krawędzi należących do rozszerzenia grafu. Dalej algorytm działa tak jak dla grafu, którego nie trzeba było rozszerzać.
\end{block}
\end{frame}

\begin{frame}{Uzasadnienie złożoności}
\begin{block}{Uwagi}
\justifying
Uzasadnienie będzie polegać na podaniu rzeczywistych kroków podczas wykonywania algorytmu i dopisywaniu po kroku jego złożoności. Przyjęto, że $G = (V,E)$, $|V| = n$, $|E| = m$.
Finalna złożoność jest łatwa do zauważenia przez wybranie największego iloczynu złożoności kroku i złożoności pętli, w których dany krok się znajduje. 
\end{block}
\end{frame}

\begin{frame}{Uzasadnienie złożoności}
\begin{block}{Znajdowanie minimalnego rozszerzenia grafu O($n^3$)}
\justifying
\begin{itemize}
    \item Wyznaczana jest liczba powtórzeń procedury budowania ścieżki. Jest to średnia ilość krawędzi wychodzących w grafie zaokrąglona w górę (dla 0 krawędzi 1) pomnożona przez współczynnik powtórzeń (retryFactor). O(n)
    \item ...
\end{itemize}
\end{block}
\end{frame}

\begin{frame}{Uzasadnienie złożoności}
\begin{block}{Znajdowanie minimalnego rozszerzenia grafu cd.}
\justifying
\begin{itemize}
    \item Wykonanie poniższej logiki tyle razy ile wynosi wyznaczona liczba powtórzeń O(n)
    \begin{itemize}
        \item Wyznaczany jest wierzchołek startowy losowo. O(1)
        \item Wykonanie procedury \textbf{podążania losową ścieżką} O($n^2$) (Wyjaśnienie poniżej)
        \item Sprawdzenie czy podążanie nową ścieżką wymagała najmniejszej ilości dodanych krawędzi do grafu. Jeśli nie to przejście do kolejnej iteracji. O(1)
        \item Przypisanie dodanych krawędzi jako najmniejszego rozszerzenia. O(1)
        \item Ustawienie cyklu tak, by zaczynał się od najmniejszego indeksu O(n)
        \item Dodanie przebytej ścieżki do zbioru cykli Hamiltona odpowiadających najmniejszemu rozszerzeniu O(1)
        \item Przerwanie pętli jeśli najmniejsze rozszerzenie ma 0 krawędzi. O(1)
    \end{itemize}
\end{itemize}
\end{block}
\end{frame}

\begin{frame}{Uzasadnienie złożoności}
\begin{block}{Podążanie losową ścieżką O($n^2$)}
\justifying
    Jest to funkcja rekurencyjna, która przyjmuje referencję do zbioru odwiedzonych wierzchołków, wierzchołek startowy, listę relacji i ilość wierzchołków grafu, zbiór dodanych krawędzi, wektor kolejno odwiedzonych wierzchołków, obecnie "odwiedzany" wierzchołek. Obecnie odwiedzany wierzchołek musi być wcześniej nieodwiedzony więc górnym ograniczeniem ilości wywołań funkcji jest ilość wierzchołków w grafie. O(n) (W wywołaniu wykonywana jest poniższa logika.) \\
    \begin{itemize}
        \item ...
    \end{itemize}
\end{block}
\end{frame}

\begin{frame}{Uzasadnienie złożoności}
\begin{block}{Podążanie losową ścieżką cd.}
\justifying
    \begin{itemize}
        \item Dodaj obecnie odwiedzany wierzchołek do zbioru odwiedzonych wierzchołków. O(1) \\
        \item Dodaj obecnie odwiedzany wierzchołek na koniec obecnie tworzonej ścieżki O(1) \\
        \item Sprawdź czy w grafie istnieje krawędź do wierzchołka startowego z "odwiedzanego" wierzchołka. O(1) \\
        \item Wybierz losowego nieodwiedzonego sąsiada odwiedzanego wierzchołka. O(n) \\
        \item Usuń krawędzie powiązane z odwiedzanym wierzchołkiem, jeśli odwiedzany wierzchołek nie jest wierzchołkiem startowym. O(n) \\
        \item ... \\
    \end{itemize}
\end{block}
\end{frame}

\begin{frame}{Uzasadnienie złożoności}
\begin{block}{Podążanie losową ścieżką cd.}
\justifying
    \begin{itemize}
        \item Jeśli nie udało się wybrać losowego sąsiedniego i nieodwiedzonego wierzchołka to wybierany jest losowy wierzchołek spośród  nieodwiedzonych, który ma najmniej krawędzi wchodzących z nieodwiedzonych wierzchołków. O(n) \\
        \item Jeśli nie udało się znaleść takiego wierzchołka to znaczy, że odwiedzono już wszyskie wierzchołki. Jeżeli brakuje krawędzi do początkowego wierzchołka to jest ona dodawana. Rekurencja się kończy. O(1)
        \item Jeśli udało się znaleść taki wierzchołek to dodawana jest krawędź od odwiedzanego wierzchołka do wybranego wierzchołka. O(1) \\
        \item Wywołana jest rekurencja dla wybranego w tym wywołaniu następnego wierzchołka. O(1)
    \end{itemize}
\end{block}
\end{frame}

\begin{frame}{Uzasadnienie złożoności}
\begin{block}{Znajdowanie cykli hamiltona w rozszerzonym grafie O($n^5$)}
\justifying
Pierwszy cykl Hamiltona był znaleziony podczas znajdowania mi minimalnego rozszerzenia grafu.
Jeśli graf nie wymaga rozszerzenia by istniał w nim cykl Hamiltona to wykonana jest poniższa procedura. 
\begin{itemize}
    \item Wyznaczona jest liczba iteracji, będąca ilością krawędzi w grafie pomnożoną przez współczynnik retryFactor. O(n)
    \item ...
\end{itemize}
\end{block}
\end{frame}

\begin{frame}{Uzasadnienie złożoności}
\begin{block}{Znajdowanie cykli hamiltona w rozszerzonym grafie cd.}
\justifying
\begin{itemize}
    \item Wykonanie poniższych kroków tyle razy ile wynosi liczba iteracji. O($n^2$) \\
    \begin{itemize}
        \item Wywołane jest podążanie losową ścieżką z wierzchołka o indeksie 0 w grafie. Różnica względem poprzedniego podążania losową ścieżką jest taka, że nie są usuwane krawędzie wychodzące z odwiedzonego wierzchołka oraz niepowodzenie przy wyborze sąsiedniego nieodwiedzonego wierzchołka kończy procedurę. O($n^2$) \\
        \item Jeśli wyznaczona ścieżka odwiedza wszystkie wierzchołki i istnieje krawędź w grafie z ostatniego do wierzchołka 0 to ścieżka ta jest dodawana do zbioru unikalnych cykli Hamiltona znalezionego rozszerzenia grafu. O(1) \\ 
    \end{itemize}
\end{itemize}
\end{block}
\end{frame}

\begin{frame}{Uzasadnienie złożoności}
\begin{block}{Znajdowanie cykli hamiltona w rozszerzonym grafie cd.}
\justifying
Do szukania pozostałych cykli jeśli minimalne rozszerzenie zawiera krawędź wykorzystana jest poniższa procedura. 
\begin{itemize}
    \item Do grafu dodane są krawędzie ze znalezionego rozszerzenia. O(n) \\
    \item Dla każdej krawędzi z rozszerzenia wykonana jest poniższa logika. O(n) \\
    \begin{itemize}
        \item Wyznaczona jest liczba iteracji będąca iloczynem retryFactor i ilości krawędzi w grafie. O(1) \\
        \item ...
    \end{itemize}
    
\end{itemize}
\end{block}
\end{frame}

\begin{frame}{Uzasadnienie złożoności}
\begin{block}{Znajdowanie cykli hamiltona w rozszerzonym grafie cd.}
\justifying
\begin{itemize}

\begin{itemize}
    \item Dla każdej iteracji wykonane są poniższe kroki. O($n^2$) \\
    \begin{itemize}
        \item Wykonane jest podążanie losową ścieżką w taki sposób jak dla przypadków, w których nie było dodatkowych krawędzi w rozszerzeniu grafu, z tym, że w pierwszym przejściu zagwarantowane jest przejście po krawędzi z rozszerzenia, a wierzchołkiem startowym jest wierzchołek z którego wychodzi krawędź z rozszerzenia grafu. O($n^2$) \\
        \item Jeśli wyznaczona ścieżka odwiedza wszystkie wierzchołki i istnieje krawędź w grafie z ostatniego do wierzchołka 0 to ścieżka ta jest dodawana do zbioru unikalnych cykli Hamiltona znalezionego rozszerzenia grafu. O(1) \\ 
    \end{itemize}
\end{itemize}
\end{itemize}
\end{block}
\end{frame}

























\end{document}
